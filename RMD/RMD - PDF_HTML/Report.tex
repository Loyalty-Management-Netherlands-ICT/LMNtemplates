% Options for packages loaded elsewhere
\PassOptionsToPackage{unicode}{hyperref}
\PassOptionsToPackage{hyphens}{url}
\PassOptionsToPackage{dvipsnames,svgnames,x11names}{xcolor}
%
\documentclass[
  10pt,
]{article}
\usepackage{amsmath,amssymb}
\usepackage{iftex}
\ifPDFTeX
  \usepackage[T1]{fontenc}
  \usepackage[utf8]{inputenc}
  \usepackage{textcomp} % provide euro and other symbols
\else % if luatex or xetex
  \usepackage{unicode-math} % this also loads fontspec
  \defaultfontfeatures{Scale=MatchLowercase}
  \defaultfontfeatures[\rmfamily]{Ligatures=TeX,Scale=1}
\fi
\usepackage{lmodern}
\ifPDFTeX\else
  % xetex/luatex font selection
\fi
% Use upquote if available, for straight quotes in verbatim environments
\IfFileExists{upquote.sty}{\usepackage{upquote}}{}
\IfFileExists{microtype.sty}{% use microtype if available
  \usepackage[]{microtype}
  \UseMicrotypeSet[protrusion]{basicmath} % disable protrusion for tt fonts
}{}
\makeatletter
\@ifundefined{KOMAClassName}{% if non-KOMA class
  \IfFileExists{parskip.sty}{%
    \usepackage{parskip}
  }{% else
    \setlength{\parindent}{0pt}
    \setlength{\parskip}{6pt plus 2pt minus 1pt}}
}{% if KOMA class
  \KOMAoptions{parskip=half}}
\makeatother
\usepackage{xcolor}
\usepackage[margin=1in]{geometry}
\usepackage{graphicx}
\makeatletter
\def\maxwidth{\ifdim\Gin@nat@width>\linewidth\linewidth\else\Gin@nat@width\fi}
\def\maxheight{\ifdim\Gin@nat@height>\textheight\textheight\else\Gin@nat@height\fi}
\makeatother
% Scale images if necessary, so that they will not overflow the page
% margins by default, and it is still possible to overwrite the defaults
% using explicit options in \includegraphics[width, height, ...]{}
\setkeys{Gin}{width=\maxwidth,height=\maxheight,keepaspectratio}
% Set default figure placement to htbp
\makeatletter
\def\fps@figure{htbp}
\makeatother
\setlength{\emergencystretch}{3em} % prevent overfull lines
\providecommand{\tightlist}{%
  \setlength{\itemsep}{0pt}\setlength{\parskip}{0pt}}
\setcounter{secnumdepth}{5}
\usepackage{fancyhdr}
\usepackage{graphicx}
\usepackage{geometry}
\usepackage{titling}
\geometry{margin=1in}

% For title page
\pretitle{%
  \begin{center}
  \LARGE
  \includegraphics[width=4cm,height=6cm]{pictures/Airmiles logo zonder achtergrond.png}\\[\bigskipamount] % Adjust the width as needed
}
\posttitle{\end{center}}

\fancypagestyle{titlepage}{
  \fancyhf{}
  \renewcommand{\headrulewidth}{0pt}
  \renewcommand{\footrulewidth}{0pt}
}

\fancypagestyle{main}{
  \fancyhf{}
  \fancyfoot[R]{\includegraphics[width=1cm]{pictures/Airmiles logo zonder achtergrond.png}} % Adjust the width as needed
  \renewcommand{\footrulewidth}{0pt}
}

\pagestyle{main} % Apply the main style to all pages
\setlength{\footskip}{10pt} % Adjust footer distance from the bottom

\AtBeginDocument{
  \thispagestyle{titlepage} % Apply the title page style to the first page
}
\usepackage{booktabs}
\usepackage{longtable}
\usepackage{array}
\usepackage{multirow}
\usepackage{wrapfig}
\usepackage{float}
\usepackage{colortbl}
\usepackage{pdflscape}
\usepackage{tabu}
\usepackage{threeparttable}
\usepackage{threeparttablex}
\usepackage[normalem]{ulem}
\usepackage{makecell}
\usepackage{xcolor}
\ifLuaTeX
  \usepackage{selnolig}  % disable illegal ligatures
\fi
\usepackage[]{natbib}
\bibliographystyle{plainnat}
\IfFileExists{bookmark.sty}{\usepackage{bookmark}}{\usepackage{hyperref}}
\IfFileExists{xurl.sty}{\usepackage{xurl}}{} % add URL line breaks if available
\urlstyle{same}
\hypersetup{
  pdftitle={Incremental Value Marketing},
  pdfauthor={Loyalty Management Netherlands},
  colorlinks=true,
  linkcolor={blue},
  filecolor={Maroon},
  citecolor={cyan},
  urlcolor={red},
  pdfcreator={LaTeX via pandoc}}

\title{Incremental Value Marketing}
\usepackage{etoolbox}
\makeatletter
\providecommand{\subtitle}[1]{% add subtitle to \maketitle
  \apptocmd{\@title}{\par {\large #1 \par}}{}{}
}
\makeatother
\subtitle{Analysing Current Trends and Market Impact}
\author{Loyalty Management Netherlands}
\date{2024-06-10}

\begin{document}
\maketitle

{
\hypersetup{linkcolor=}
\setcounter{tocdepth}{2}
\tableofcontents
}
\hfill\break

\newpage

\hypertarget{management-summary}{%
\section{Management Summary}\label{management-summary}}

blabla

\newpage

\hypertarget{introduction}{%
\section{Introduction}\label{introduction}}

In today's fast-paced digital landscape, where consumer behaviour and
market dynamics are continuously evolving, it is imperative for
businesses to adopt a data-driven approach to marketing. At Air Miles,
we recognise the importance of leveraging advanced analytical techniques
to optimise our strategies and deliver superior value to our partners
and shareholders. One such technique that stands out in its efficacy and
precision is the holdout test, particularly within the realm of email
marketing.

Air Miles currently has \textbf{3145519} active customers of which
\textbf{2024028} are opt-in. We sent 100 million emails per year to our
customers but we currently do not know the results of this effort on the
customer behaviour to redeem or save points at our partners.

\hypertarget{what-is-a-holdout-test}{%
\subsection{What is a Holdout Test}\label{what-is-a-holdout-test}}

A holdout test, also known as an A/B test or control experiment, is a
powerful method used to evaluate the incremental impact of email
marketing. It involves creating two distinct groups from the our member
base: a test group and a holdout (control) group. The test group
receives the marketing intervention---in this case, email marketing
campaigns--- while the holdout group does not. By comparing the outcomes
between these two groups, we can isolate the impact of our email
marketing efforts and gain valuable insights into their true
effectiveness.

We started this experiment on the first of April until the end of June.
In the test only commerical emails were included. Email such as surveys,
onboarding flow for the first six weeks and service emails (expired Air
Miles, forgot password, etc.) were still sent to the entire base.

\hypertarget{impact-on-customer-behaviour}{%
\section{Impact on Customer
Behaviour}\label{impact-on-customer-behaviour}}

\begin{itemize}
\tightlist
\item
  (+x\% repurchases\& x\% redemptions) + graph split per Partner (AHold,
  Shell, Praxis) Conclusie: We see a significant uplift in repurchases
  for Shell \& Praxis, but not for AH (check Do we see a negative impact
  on AH transactions? Specific redemptions)
\end{itemize}

\hypertarget{impact-per-partner-shell}{%
\section{Impact per Partner: Shell}\label{impact-per-partner-shell}}

\hypertarget{savings}{%
\subsection{Savings}\label{savings}}

+X Million euro due to marketing

\begin{itemize}
\item
  Shopaankopen - Barchart (colors Ritmes) Conclusie: We see an uplift in
  X\% shop transactions, most loyal customers have the biggest uplift
  Resulting in 1Million Euro more turnover
\item
  Liters+ Fuel visits -- Barchart (colors Ritmes) Conclusie: We see an
  uplift in X\% Fuel transactions, least loyal customers have the
  biggest uplift Resulting in 10Million Euro more turnover
\end{itemize}

\hypertarget{redemptions}{%
\subsection{Redemptions}\label{redemptions}}

\begin{itemize}
\tightlist
\item
  Redemptions - Barchart (colors Ritmes) Conclusie: We see an uplift in
  X\% Redemption transactions, most loyal customers have the biggest
  uplift We can convince most loyal shell customers to spent their
  earned points at Shell
\end{itemize}

\hypertarget{app-downloads}{%
\subsection{App Downloads}\label{app-downloads}}

App Downloads + X\% more App downloads - Focus on which Ritme

\hypertarget{email-campaigns-with-largest-impact}{%
\subsection{Email Campaigns with Largest
Impact}\label{email-campaigns-with-largest-impact}}

\begin{itemize}
\tightlist
\item
  Which Emails have the biggest impact?
\end{itemize}

\hypertarget{impact-per-partner-praxis}{%
\section{Impact per Partner: Praxis}\label{impact-per-partner-praxis}}

\hypertarget{savings-1}{%
\subsection{Savings}\label{savings-1}}

\begin{itemize}
\tightlist
\item
  Shop visits -- BarChart (Praxis Loyal customers vs non-loyal
  customers) We see an uplift in X\% Shoptransactions, least loyal
  customers have the biggest uplift Resulting in 10Million Euro more
  turnover
\end{itemize}

\hypertarget{redemptions-1}{%
\subsection{Redemptions}\label{redemptions-1}}

\begin{itemize}
\tightlist
\item
  Barchart (Praxis Loyal customers vs non-loyal customers)
\end{itemize}

\hypertarget{email-campaigns-with-largest-impact-1}{%
\subsection{Email Campaigns with Largest
Impact}\label{email-campaigns-with-largest-impact-1}}

\begin{itemize}
\tightlist
\item
  Which Emails have the biggest impact on Value creation for Praxis
\end{itemize}

\end{document}
